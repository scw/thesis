% what does it mean? Interpret the patterns observed within the data. Link back to a broader discussion.

Shipping is a major user of the ocean, but little is known about its distribution and effects. Here I built the first validated and global models of ship movement. Shipping companies acknowledge the importance of managing the ocean holistically, but lack the scientific knowledge and tools to do so effectively. By incorporating ecological information alongside logistical efficiencies, system robustness can be improved.

\section{Protected Areas and Spatial Planning}

One approach for incorporating ecological information in ocean decision making is with marine spatial planning (MSP), a promising avenue for bringing stakeholders together~\citep{merrifield2012marinemap}. This approach requires extensive data, particularly on the three major areas of use: fisheries management, transportation, and energy management~\citep{Lubchenco2010}. Transportation in the ocean is the least studied of the three, and here I have shown how volunteered geographic information methods, along with volunteered observations, can provide us a way to approach the data-poor problem of shipping.

% XXX MPAS, LIMITS DYNAMICS
Marine protected areas (MPAs) have been shown to be effective~\citep{halpern2002marine}, however multi-dimensional ocean use can also take advantage of dynamic MPAs, to mitigate against infrequent but high-risk events~\citep{Boersma1999}. Dynamic MPAs may also be introduced to manage seasonal events, such as species migrations, in a way that shipping lanes and general speed reductions can not accommodate. % Looks good. May be worth looking at things like Boersma and Parrish, 1999 "Limiting Abuse: marine protected areas, a limited solution" to get a sense of how cumulative risk can still mean that low probability events can be worth paying attention to in this space. But definitely: improved regulation, addition of MPAs, introduction of AIS have all helped limit exposure to this risk.
Dynamic MPAs will likely rely on providing users, such as ship operators, with real-time information about the state of the environment. Incorporating environmental data directly within AIS systems~\citep{2011AGUFMIN11B1278S} provides a useful way to integrate across these domains.

\section{Ecological Effects}

The ecological effects of shipping have been touched on in this manuscript, an example of ecology coming to terms with the spatial context at the unit of analysis, allowing model scope to extended across disciplines~\citep{tilman1997}. A few illustrations of how this work can be applied to ecological questions follow.

\paragraph{Ballast} Ballast water is used to prevent vessel capsizing by weighting the ship within the water, vessels can transport upwards of 20 million gallons of water for ballast. By transporting large parcels of water between otherwise disconnected areas of the ocean, ballast water introduces ecological coupling and can transport invasive species~\citep{Ruiz2000,Keller2011}. Ballast water exchange also can transport viruses and bacteria, which have human health consequences beyond ecosystem function. By providing global, per-vessel movement, ballast exchange can be examined in greater detail, as can vessel use of ballast exchange points. 

\paragraph{Ship Strikes} Ship strikes, or species to vessel collisions, also cause injury and death to marine mammals, driving whale population declines, as is the case with the North Atlantic right whale~\citep{Moore2007}. Globally, shipping increased three-fold in the last 50 years, coinciding with increases in average ship speed \citep{Vanderlaan2009}. This combination has led led to large increases in shipping-induced marine mammal fatalities. The IMO has helped implement improvements to shipping lanes~\citep{Lagueux2011,Mckenna2012a}, but the model produced here can help us in locating key whale-vessel areas in those locations not as intensively monitored as the US coastal waters. Dynamic MPAs may also serve a role to mitigate this impact.

\paragraph{Noise Pollution} Sound, which is efficiently transmitted through water, is used by many species who have adapted sensitive hearing to forage and communicate. Thus, engine noise from ships has led to extensive sound pollution, causing behavioral changes in a variety of marine species, and mortality when exposed to loud sounds from naval activities~\citep{Hatch2009}. NOAA recently created a working group to evaluate the impact of noise pollution on marine life, and the model created here improves our understanding of the distribution of shipping, which can be incorporated into future models of ship noise to improve ocean management.

% XXX REGULATION AS IMPORTANT SECTION
\section{Regulation}

The primary extractive use of the ocean is fishing, where our understanding has been advanced by intensive investments to study its effects. % XXX citation for this? its clear from spending, but quantify it.
Many fisheries are now managed systems~\citep{worm2009rebuilding}, as unmanaged fish stocks have repeatedly collapsed~\citep{costello2012status}. % clearly disagreement in the Worm / Jackson / Halpern vs. Hilborn camps, use their joint papers as an end-run around this.
Similar management and regulation is noticeably absent from the shipping industry.
  
Shipping is regulated by the International Maritime Organization (IMO), which enforces industry rules and standards. However, the transnational nature of the industry has led to limited enforcement. As is common in consensus-based international bodies~\citep{cogan2009representation}, the IMO can be slow to respond to issues.
  % XXX have to cite both of these statements if this is gonna stick. INSTEAD: say that they are a CONSENSUS based organization which requires ratification of many international bodies, by its nature a time consuming process. 
  For example, while the IMO acknowledges that a 20\% reduction in greenhouse gasses can be accomplished this decade without additional costs to operators \citep{imo2009}, it has not implemented new emissions rules.  In light of the regulatory environment, alternatives, such as tying ship insurance rates to ecologically important areas of the ocean, may be preferable.
 % an example article on IMO problems: http://articles.cnn.com/2012-07-04/world/world_europe_costa-concordia_1_concordia-disaster-cruise-industry-cruise-ship-disaster/3?_s=PM:EUROPE
 % http://ban.org/ban_news/2006/060316_imo_treaty.html
 % http://www.sourcewatch.org/index.php?title=International_Maritime_Organization#Criticisms_of_the_IMO

%   + other parts of the ocean environment coming into regulation, particularly fishing
%   + Regulation of shipping rudimentary
%    * wild west in terms of control; do regulate emissions in specific places but international agreements minimal (what's IMO got on the table)
%    * part of the problem is that shipping is extremely international, more so than fishing
%    * primary regulation is at ports, otherwise just GPS units and a hope that no one crashes

% XXX XXX XXX

\section{Transportation Networks}

%Some analyses require "factoring out" geography to the extent possible, such as ecological models (Tilman book), Economic, et cetera. Network theory provides a useful basis of analysing geographically explicit data in a mathematical framework which can incorporate some of the geographic constraints while eliding many details necessary in a spatially-explicit model.

% another pp from a position
% This becomes particular important as the tools of spatial thinking extend across disciplinary boundaries, both in the social and physical sciences (Goodchild and Janelle, 2010; Tilman and Kareiva, 1997). Other domains, such as ecology and economics, are coming to terms with the fact that their historical approaches to keeping simple models (and by extension, limiting model scope to their domain) ignores important spatial context which naturally arises at the unit of analysis (Tilman and Kareiva, 1997; Krugman, 1991).

% XXX Oliver: By now, I've lost track of why I should care about transportation networks.  How do transportation networks affect the things you've discussed in the previous section?  Is there something inherent in their nature or structure that's particularly relevant?  If you're just pointing out that there's not a lot of useful data available, is it worth it's own section?

One approach to managing a complex system like shipping is to study its network representation.  Transportation networks with geographically-fixed edges and nodes, such as road and rail, form the basis of transportation geography \citep{Rodrigue2009}. Unlike land-based networks, shipping exists in between the extremes of a fixed network % XXX DEFINE, Newman 2010?
 and two dimensional Brownian bridges, as vessels must transport both goods and passengers to specific locations (primarily ports) but are free in choosing movement between destinations, except in near shore areas regulated by shipping lanes. Air transportation shares network similarities, but strong regulation has led to predetermined flight paths, with deviations generally limited to emergencies or extreme weather. % XXX need a ref here. how does the FAA and the relevant international body regulate movement? XXX XXX find something, in transportation geography book? where?

Because of inherent risks in air transportation, detailed, well-vetted information on flight paths is provided by government agencies \citep{guimera2005worldwide}, simplifying modeling. Ship movements have no equivalent top-down data collection effort, and while organizations such as the US Coast Guard have made efforts, the system remains rudimentary. Most information is provided via private contracts, through organisations such as Lloyd's of London, who have been recording information on shipping since 1774 \citep{Lloyd'sRegister-FairplayLtd.2010}. As a result, limited public data are available on the shipping trade, despite being identified by the Federal Geographic Data Committee~\citep{FGDC} as a key transportation component to the national spatial data infrastructure \citep{CurrierInPress}.

Transportation networks play an important economic role \citep{canning1993effects}. Because this role, information on shipping is valuable, leading to a cottage industry of information sales and limiting its public availability.  A future extension of this work might incorporate network theory, which has been applied to port-to-port networks~\citep{Kaluza2010,ducruet2012worldwide}, but has not yet been applied to spatially explicit vessel movement.

% TALK ABOUT OTHER MODES OF TRANSPORTATION; [this gives you a way to tame it, its reasonable to assume that it'll map to one of these models]

% - shipping has big economic benefit, but costs are externalized
% - there are good actors in shipping who want to do the right thing, but don't have the tools to do it [cites, +Maersk lady]
% - only scientific community can do this
% - can only do msp by bringing stakeholders together
% - VGI methods provide good way of handling messy data as we try to move it toward 'reality'
% - coupling these approaches with real-time efforts has potential to shift dynamic.

% TODO Goodchild: there's an opportunity to attach probabilities to individual points based on other points in the track.

% XXX BONUS STUFF ON LINKAGES TO OTHER FIELDS
%Other domains, such as ecology and economics, are coming to terms with the fact that their historical approaches to keeping simple models (and by extension, limiting model scope to their domain) ignores important spatial context which naturally arises at the unit of analysis \citep{tilman1997,krugman1991geography}.
