% from Oliver: WHY THE OCEANS ARE AWESOME, WHAT ADDRESSING SHIPPING DOES TO THE OCEANS, AND HOW KNOWING MORE WILL HELP SPECIFICALLY ADDRESS THESE ISSUES, with citations.

% - The importance of shipping means that data related to shipping is valuable
% - Leaves a gap in the data picture for scientific inquiry

% FRAME THE ISSUE BROADLY
% Shipping, the global ocean transportation of people and goods, provides a useful window into critical issues for geographic information science (GIScience), and its understanding has implications for a variety of applied fields including marine conservation, marine spatial planning, transportation geography, and navigation.  This work examines a synthesis of available shipping data.

The ocean is integral to supporting life, by producing much of the planet's biomass, acting as a critical step in the hydrologic cycle, and by regulating global climate. Oceans also support human needs by providing nutrition, resources, and an efficient medium for transportation. However, a confluence of changes is compromising the ocean's ability to support both marine life and meet human needs, with a recent study identifying climate change, fishing, and shipping as the greatest anthropogenic threats to marine health~\citep{Halpern2008}.  Shipping, the ocean transportation of people and goods, moves \$1.8 trillion dollars of trade annually, constitutes a \$300 billion dollar maritime industry, and plays a crucial role in the global transportation network, where it conveys 90 \% of world goods~\citep{oced2010,Rodrigue2009}. Preservation of the ocean's benefits calls for holistic management~\citep{Lubchenco2010}, which in turn requires detailed quantitative information. Current efforts have emphasized broad abiotic factors such as climate, and biotic factors such as fishing, but little work has examined the role of shipping, despite its importance to both the economy and the environment. This manuscript seeks to bridge this gap, by expanding our knowledge of shipping to address issues of marine conservation, marine spatial planning, transportation geography, and navigation.

%Shipping is also important in a broader context, and as we move toward the holistic management of the oceans, foundational data on human use is necessary~\citep{Lubchenco2010}. 

% EXPLAIN WHAT IS AND IS NOT KNOWN
% what we do know:
%\section{Existing Work}
Shipping produces many ecological impacts, but the effects largely remain out of sight, and little research looks beyond local scales. Ecological effects which have been examined include groundings, which cause direct effects like oil spills and habitat destruction, such as the \$3 billion Exxon-Valdez spill\footnote{The \$5 billion liability for the spill was so great that a now infamous financial instrument was manufactured to absorb it: the credit default swap.}.  Shipping contributes to climate change, where greenhouse gas emissions of the industry accounts for five percent of total man-made emissions~\citep{Eyring2009}, and its pollution causes 60,000 premature deaths per year~\citep{Corbett2007}.  Ship strikes also cause injury and death to marine mammals, and have driven whale population declines, for example the North Atlantic right whale (\textit{Eubalaena glacialis}), where human activity accounts for half of all fatalities, primarily caused by ship strikes and gear entanglement~\citep{Moore2007}. Globally, shipping has increased three-fold in the last 50 years, coinciding with increases in average vessel speed~\citep{Vanderlaan2009}. This combination has led led to large increases in shipping-induced marine mammal fatalities, despite efforts to track ship-whale interactions \citep{jensen2004large} and improvements to shipping lanes~\citep{Lagueux2011,Mckenna2012a}.

Ship ballast water, used to control vessel stability, also transports invasive species long distances, leading to widespread economic and biological losses~\citep{Ruiz2000,Rodrigue2009}. Transported species include plants, animals, bacteria, and human pathogens, resulting in major changes to many nearshore systems, particularly coastal lagoons and inlets~\citep{leppakoski2002baltic}.  Sound, which is efficiently transmitted by water, is used by manymarine species who have adapted sensitive hearing to forage and communicate. Engine noise from ships has led to extensive noise pollution, causing behavioral changes in a variety of marine species, and the loud sounds of naval activities can cause mortalities~\citep{Hatch2009}. With an improved understanding of shipping, further ecological and human impacts might also be identified.

% Other ecological and human impacts no doubt exist, but the poor understanding of shipping has left them unidentified~\citep{Davenport2006}.

% Oliver: Either I don't understand the however, or I don't understand what you mean by global phenomenon.  Were previous studies primarily local?  Are you going to integrate what previous studies have done?  Do you mean specific geographic areas, or specific areas within the limited field of shipping studies?
Little research has evaluated shipping as a global, interconnected system, where vessels move between a complex network of ports spanning the globe. The first work to examine the global shipping trade in an ecological context was \cite{Halpern2008}, a synthesis effort to collect information across habitat types and impacts. The work evaluates many human uses of the ocean, and gathers together global data to examine the cumulative ocean impact of humans, ranging from changes in sea surface temperature due to radiative forcing, to land-based nutrient runoff. The data used in this analysis to represent shipping, however, has limitations: It ignores vessel type, % TODO this is the first point at which I mention ship classes, what shipping *IS* should be explained earlier in the document.
 and contains information on only 12\% of the fleet. The ships it does include are also a spatially- and statistically-biased sample of the population \citep{Wang2007}, and the data vary between vessels, leaving many areas with known use missing from the model. Finally, the paper does not tackle modeling issues, but instead provides patterns without the context necessary to address deeper questions.

Another paper examining global shipping, \cite{Kaluza2010} instead focuses on the networks contained within shipping, providing an abstract network model of ship movement. Licensing data from Lloyd's Register Fairplay, % according to 'Data' section of Kalusa2010
 the authors aggregate port of call records, which are sequential lists of vessel location. They then extrapolate port-to-port links into routes, by combining land-based barriers with great circle distance calculations to select the path of a ship between two ports. This novel approach provides a measure of connectivity between port locations, and improves on the gravity model widely used to predict connectivity.  However, their work provides insufficient detail to extract true ship movement paths, a necessary step for effective spatial management. Instead, geographically referenced facts about movement are necessary. The data used also requires expensive licensing, making any extension of the work both financially prohibitive and limiting its accessibility. Despite largely ignoring geography, \cite{Kaluza2010} is currently the best examination of the global shipping trade.

% Much research focuses on specific areas within shipping, such as maritime awareness to detect anomalous behavior \citep{Tun2007}, tracking ship-whale interactions \citep{jensen2004large}, or shipping lanes to protect a species \citep{Lagueux2011,Mckenna2012a}.  Few papers tackle shipping as a global, interconnected phenomena, where vessels move between a complex network of ports spanning the globe. 

% ESTABLISH WHY THE QUESTION I HAVE IS AN IMPORTANT ONE

% due in part to the poor light penetration in water, limiting the effectiveness of remote sensing~

As we try to understand the health of the ocean \citep{Halpern2012}, we require quantitative data on the marine environment and its human use, both of which pose acute challenges to ocean scientists, due to our limited means to capture information in both the open ocean and at depth~\citep{Wright1997}. These data are necessary to inform decision-making, and are also important to minimizing the regulatory burden on the shipping industry. By managing shipping as a complex system, efficiency can be tied to improved environmental conditions. This work contributes foundational knowledge on the state and distribution of shipping, which can help understand this key user of the ocean, and identify key areas to examine for mitigating ecological impact.

%\section{Where this work comes in}

% SPELL OUT WHAT THIS STUDY DOES

% Halpern: This section jumps between many different topics and doesn't really explain them well enough to understand.  You either need to expand this a lot, or cut it way down and just state what the purpose of this study is.

% GIScience seeks to formalize our understanding of geographic information, and shipping can provide insight into areas of the GIScience agenda. 
% THIS IS GOING TO BE THE FIRST WORK TO *SHARE* DATA ON THE GLOBAL DISTRIBUTION OF SHIPPING, ALONG WITH DETAILED ATTRIBUTES NECESSARY TO DIG INTO THE TRICKY QUESTIONS WE HAVE IN ECOLOGICAL ISSUES.

% Attributes useful for ecological questions:
%    Noise: engine type, length, type
%    Ballast: type, length
%    Strikes: Maximum speed, type, draft

This is the first work to combine extensive global ship observations~\citep{VOSClim,Tetreault2002} with vessel identification records to both address questions on ecological effects of shipping, and to expand our understanding of the global shiping trade. Most previous works focused on regional problems, and the limited global work available used poor data~\citep{Corbett2007, Halpern2008}. Incorporating geographic quality checking methods~\citep{goodchildli2012} with volunteered geographic information (VGI, \citealp{goodchild2007citizens}), I provide a spatially resolved high resolution dataset which links individual ships with their movement patterns and vessel attributes. This dataset, when aggregated into multiple representations, allows us to answer open questions on topics including noise pollution, ship strikes, and invasive species introduction. 

% By fusing together data from multiple sources, here I build a dataset useful for interpreting the global shipping trade. While high resolution (spatially and temporally) volunteered information exists, its use has been limited to regional problems, and existing global work used poor data~\citep{Corbett2007, Halpern2008}. This paper contributes data predominantly within 100km of shores, where most human and biological users of the ocean persist, and building our knowledge of these areas is particularly valuable. Ship traffic is most dense, regulated, and complex within the exclusive economic zones of nations, and it is useful property that this data are dense within these areas.




% ... old stuff below

% Shipping places a growing demand on the ocean, transporting \$1.8 trillion dollars of goods annually, and providing a \$300 billion dollar maritime industry, representing one of the largest economic users of the ocean. This activity impacts both human health and the environment, such as the \$3 billion Exxon-Valdez spill\footnote{The \$5 billion liability for the spill was so great that a now infamous financial instrument was manufactured to absorb it: the credit default swap.}, but largely the effects of shipping remain out of sight. Documented effects include greenhouse gasses, accounting for five percent of total man-made emissions, and causing 60,000 premature deaths per year \citep{Corbett2007}. Ecologically, vessels cause ship strikes which can impair whale populations \citep{Fujiwara2001}, % XXX better refs? 
% transport invasive species, leading to widespread economic and biological losses, cause groundings with a host of direct effects like oil spills and habitat destruction, and produce noise pollution, leading to mortality events. Other ecological and human impacts are likely to exist, but remain unidentified, as shipping continues to be poorly understood~\citep{Davenport2006}. As we try to understand the health of the ocean \citep{Halpern2012}, we need detailed data, which poses an acute challenge to ocean scientists~\citep{Wright1997}. These data are necessary to inform decision-making, such as marine spatial planning, but is also important to minimizing the regulatory burden on the shipping industry: by managing shipping as a complex system, efficiency can be tied to improved environmental conditions.

% - Kaluza paper, based on Lloyd's data
%   + written for network theory audience
%   + data is port order for specific ships, which is used to come up with proposed routes of ships based on great circle distances (A to B to C)  
%   + problem - they don't actually know where ships travelled, which means results aren't helpful for problems where ship locations actually matter.
%   + can't share data, it's proprietary
%   + this paper is still the best that looks at shipping at a global level
%   + ignores geography

% - Work on global shipping
%   + Lots of papers that focus on specific parts of the shipping problem - few that take large perspective.
%   + Ben's Science paper
%    * first work that looked at global impact of shipping from ecological context, but wasn't focused on shipping
%    * first work to really look at how humans are affecting marine environment by synthesizing together all available global data
%    * problem with data is that it was based upon unrepresentative ships - missing whole categories of ships; only 12\% of fleet; representation issues (see Corbett \& Wang paper for details)
%    * second problem, uneven spatial distribution - tons of gaps. Data quality highly variable from ship to ship
%     - missing big areas of the ocean that we know lots of ships are using
%    * problem three - doesn't address data models (important for geographers); just says 'ships are here'.

% Moved this paragraph to CONCLUSION
% As noted by Goodchild, ``changing technology and economics are moving map production from a system of unified central production to a local patchwork, and the old radial system of dissemination is being replaced with a complex network''\citep{goodchild1999cartographic}. By using quality assurance methods from both computer science and geography, such as record linage and geographic validation, I can filter unreliable inputs. The near future will involve global, real-time, high resolution ship data \citep{JonesGoogle2012,carson2012satellite}, but I continue to need methods which accommodate data curation and integration, and allow us to address specific hypotheses. This is a key appeal of volunteered information: some problems of uncertainty become tractable with sufficient observation volume, as we can evaluate the distribution directly instead of relying on sampling methods.
