\section{Shipping, who cares?}
% XXX REWRITE THIS FIRST PART: NEEDS A PUNCH TO LET PEOPLE KNOW WHY THIS MATTERS.
% XXX XXX ADD, from Oliver: WHY THE OCEANS ARE AWESOME,  WHAT ADDRESSING SHIPPING DOES TO THE OCEANS, AND HOW KNOWING MORE WILL HELP SPECIFICALLY ADDRESS THESE ISSUES, with citations.
Shipping, the global transportation of people and goods, provides a useful window into critical issues for geographic information science (GIScience), and its understanding has implications for a variety of applied fields including marine conservation, marine spatial planning, transportation geography, and navigation.  This work examines a synthesis of available shipping data.

Shipping places a growing demand on the ocean, transporting \$1.8 trillion dollars of goods annually, %XXX need reference! 
and providing a \$300 billion dollar maritime industry, representing one of the largest economic users of the ocean. This activity impacts both human health and the environment, such as the \$3 billion Exxon-Valdez spill\footnote{The \$5 billion liability for the spill was so great that a now infamous financial instrument was manufactured to absorb it: the credit default swap.}, but largely the effects of shipping remain out of sight. Documented effects include greenhouse gasses, accounting for five percent of total man-made emissions, and causing 60,000 premature deaths per year \citep{Corbett2007}. Ecologically, vessels cause ship strikes which can impair whale populations \citep{Fujiwara2001}, % XXX better refs? 
 transport invasive species, leading to widespread economic and biological losses, cause groundings with a host of direct effects like oil spills and habitat destruction, and produce noise pollution, leading to mortality events. Other ecological and human impacts are likely to exist, but remain unidentified, as shipping continues to be poorly understood~\citep{Davenport2006}. As we try to understand the health of the ocean \citep{Halpern2012}, we need detailed data, which poses an acute challenge to ocean scientists~\citep{Wright1997}. These data are necessary to inform decision-making, such as marine spatial planning, but is also important to minimizing the regulatory burden on the shipping industry: by managing shipping as a complex system, efficiency can be tied to improved environmental conditions.

GIScience seeks to formalize our understanding of geographic information, and shipping can provide insight into areas of the GIScience agenda. Ship observations are contributed by vessels that submit data on both a voluntarily and mandated basis \citep{VOSClim,Tetreault2002}, affording an opportunity to advance our use of geographic quality checking methods \citep{goodchildli2012} against this volunteered geographic information (VGI) \citep{goodchild2007citizens}. And while individual observations have a simple representation (a location, time, and attributes), effective use requires multiple representations \citep{Goodchild1992} and the spatio-temporal modeling techniques of time geography \citep{miller2008field}.

%   + invasive species via ballast water 
%   + ship strikes
%   + noise pollution
%   + groundings (direct effects like oil spills, grey water, chemical spills, smashing coral)
%    * acute damage
%    * chronic damage
%   + certain to be many other ecological and human impacts -- just a poorly understood use of the ocean.

% XXX what are the economic uses mentioned in the OHI paper? put this in context.
% XXX from Oliver: This feels like a non-sequitur.  Fishing is fundamentally an extractive industry.  Shipping is transportation.  Does fisheries management help us manage shipping?
A major extractive user of the ocean is the fishing industry, where our understanding has been advanced by intensive investments to study its effects. % XXX citation for this? its clear from spending, but quantify it.
Many fisheries are now managed systems~\citep{worm2009rebuilding}, in part due to frequent \textit{tragedy of the commons} events, leading to the collapse of unmanaged fish stocks~\citep{costello2012status}. % clearly disagreement in the Worm / Jackson / Halpern vs. Hilborn camps, use their joint papers as an end-run around this.
Similar management and regulation is noticeably absent from the shipping industry.
% XXX oliver: JUMPED FROM FISHING TO INSURANCE, DO I ACTUALLY DO ANYTHING WITH THIS STUFF? Not really. ?bro
  Shipping is regulated by the International Maritime Organization (IMO), a branch of the United Nations which enforces rules and standards on the industry. However, the transnational nature of the industry has led to low enforcement rates. As is common in consensus-based international bodies~\citep{cogan2009representation}, the IMO is slow to respond to problems of the day. 
  % XXX have to cite both of these statements if this is gonna stick. INSTEAD: say that they are a CONSENSUS based organization which requires ratification of many international bodies, by its nature a time consuming process. 
  For example, while acknowledging that 20\% reductions in greenhouse gasses can be accomplished in this decade, without additional costs to operators \citep{imo2009}, it has not implemented new rules on such gasses.  In light of the weak regulatory environment, alternatives, such as tying ship insurance rates to specific areas of the ocean, may be preferable.
 % an example article on IMO problems: http://articles.cnn.com/2012-07-04/world/world_europe_costa-concordia_1_concordia-disaster-cruise-industry-cruise-ship-disaster/3?_s=PM:EUROPE
 % http://ban.org/ban_news/2006/060316_imo_treaty.html
 % http://www.sourcewatch.org/index.php?title=International_Maritime_Organization#Criticisms_of_the_IMO

%   + other parts of the ocean environment coming into regulation, particularly fishing
%   + Regulation of shipping rudimentary
%    * wild west in terms of control; do regulate emissions in specific places but international agreements minimal (what's IMO got on the table)
%    * part of the problem is that shipping is extremely international, more so than fishing
%    * primary regulation is at ports, otherwise just GPS units and a hope that no one crashes

% FUNDAMENTALLY, doing synthesis on the shipping, with a set of simple rules to validate observations.

% XXX XXX Oliver: By now, I've lost track of why I should care about transportation networks.  How do transportation networks affect the things you've discussed in the previous section?  Is there something inherent in their nature or structure that's particularly relevant?  If you're just pointing out that there's not a lot of useful data available, is it worth it's own section?

%\subsection{Transportation networks}
One approach to managing a complex system like shipping is to study its network representation.  Transportation networks with geographically-fixed edges and nodes, such as road and rail, form the basis of transportation geography \citep{Rodrigue2009}. Unlike these networks, shipping exists in between the extremes of a fixed network % XXX DEFINE, Newman 2010?
 and two dimensional Brownian bridges, as vessels must transport both goods and passengers to specific locations (primarily ports) but are free in choosing movement between destinations, except in near shore areas regulated by shipping lanes. Air transportation shares network similarities, but strong regulation has led to predetermined flight paths, with deviations generally limited to emergencies or extreme weather. % XXX need a ref here. how does the FAA and the relevant international body regulate movement?

Because of inherent risks in air transportation, detailed, well-vetted information on flight paths is provided by government agencies \citep{guimera2005worldwide}, simplifying modeling. Ship movements have no equivalent top-down data collection effort, and while organizations such as the US Coast Guard have made efforts, the system remains rudimentary. Most information is provided via private contracts, through organisations such as Lloyd's of London, who have been recording information on shipping since 1774 \citep{Lloyd'sRegister-FairplayLtd.2010}. As a result, limited public data is available on the shipping trade, despite being identified by the Federal Geographic Data Committee~\citep{FGDC} as a key transportation component to the national spatial data infrastructure \citep{CurrierInPress}.

% XXX where are you going with this?
Transportation networks play an important economic role \citep{canning1993effects}. Because this role, information on shipping is valuable, leading to a cottage industry of information sales and limiting its public availability. Here, we look at using geographically-volunteered information to infer the movement patterns which in part define the shipping network.

% XXX XXX XXX
% EXPAND THIS SECTION TO COMPARE BETWEEN NETWORK TYPES
% TALK ABOUT OTHER MODES OF TRANSPORTATION; [this gives you a way to tame it, its reasonable to assume that it'll map to one of these models]
\section{Existing work}
uch research has focused on specific areas within shipping, such as maritime awareness to detect anomalous behavior \citep{Tun2007}, tracking ship-whale interactions \citep{jensen2004large}, or shipping lanes to protect a species \citep{Lagueux2011,Mckenna2012a}.  Few papers, however, tackle shipping as a global phenomena. Two important papers from this limited body of research will now be discussed.
% XXX XXX Oliver: Either I don't understand the however, or I don't understand what you mean by global phenomenon.  Were previous studies primarily local?  Are you going to integrate what previous studies have done?  Do you mean specific geographic areas, or specific areas within the limited field of shipping studies?

The first work % XXX can I say this really?
to examine the global shipping trade in an ecological context was a synthesis effort to collect information across habitat types and impacts \citep{Halpern2008}. The work evaluates many human uses of the ocean, and gathers together global data to examine the cumulative impact humans have, ranging from changes in sea surface temperature due to radiative forcing, to land-based nutrient runoff. The data used in this analysis to represent shipping, however, has limitations: it ignores vessel classes, % XXX this is the first point at which I mention ship classes, what shipping *IS* should be explained earlier in the document.
 and contains information on only 12\% of the fleet. The ships it does include are a spatially- and statistically-biased sample of the population \citep{Wang2007}. The data is variable between ships, so many areas where there is active use are missing from the model. Finally, the paper does not tackle modeling issues, but instead provides patterns without the context necessary to address deeper questions. % XXX cite the SOM? is that needed?

% - Work on global shipping
%   + Lots of papers that focus on specific parts of the shipping problem - few that take large perspective.
%   + Ben's Science paper
%    * first work that looked at global impact of shipping from ecological context, but wasn't focused on shipping
%    * first work to really look at how humans are affecting marine environment by synthesizing together all available global data
%    * problem with data is that it was based upon unrepresentative ships - missing whole categories of ships; only 12\% of fleet; representation issues (see Corbett \& Wang paper for details)
%    * second problem, uneven spatial distribution - tons of gaps. Data quality highly variable from ship to ship
%     - missing big areas of the ocean that we know lots of ships are using
%    * problem three - doesn't address data models (important for geographers); just says 'ships are here'.

A more recent paper \citep{Kaluza2010} takes a different approach, by focusing on shipping as a network, with an analysis geared toward a network theory audience. Licensing data from Lloyd's of London, % XXX cite?
 the authors aggregate port of call records, which are sequential lists of vessel location. They then extrapolate port to port links into routes, by combining land-based barriers with great circle distance calculations to select the 'path' of a ship between two ports. This novel approach provides a measure of connectivity between port locations, and improves on the gravity model widely used to predict connectivity. % XXX cite
However, this model provides insufficient detail to extract ship movement paths, necessary for resolving many of the conflicts listed above. Instead, we need geographically referenced facts about movement. Additionally, the data used requires expensive licensing, making extending the work both financially prohibitive and limiting its accessibility. Despite largely ignoring geography, this paper is currently the best examination of the global shipping trade.

% - Kaluza paper, based on Lloyd's data
%   + written for network theory audience
%   + data is port order for specific ships, which is used to come up with proposed routes of ships based on great circle distances (A to B to C)  
%   + problem - they don't actually know where ships travelled, which means results aren't helpful for problems where ship locations actually matter.
%   + can't share data, it's proprietary
%   + this paper is still the best that looks at shipping at a global level
%   + ignores geography

\section{Where this work comes in}

Fusing together data from multiple sources, here I build a dataset useful for interpreting the global shipping trade. While high-resolution (spatially and temporally) volunteered information exists, its use has been limited to regional problems, and existing global work used poor data~\citep{Corbett2007, Halpern2008}. This paper contributes data predominantly within 100km of shores, where most human and biological users of the ocean persist, and building our knowledge of these areas is particularly valuable. Ship traffic is most dense, regulated, and complex within the exclusive economic zones of nations, and it is useful property that this data is dense within these areas.
% XXX oliver: It might be worth expanding on "where most human and biological users of the ocean persist".  It's worth emphasizing that most of the stuff we care about occurs in this region, which is also where ship traffic is the most common, the hardest to predict (lots of ships of all sorts, lots of destinations), and the most regulated (shipping lanes).

As noted by Goodchild, ``changing technology and economics are moving map production from a system of unified central production to a local patchwork, and the old radial system of dissemination is being replaced with a complex network''\citep{goodchild1999cartographic}. By using quality assurance methods from both computer science and geography, such as record linage and geographic validation, we can filter unreliable inputs. The near future will involve global, real-time, high resolution ship data \citep{JonesGoogle2012,carson2012satellite}, but we continue to need methods which accommodate data curation and integration, and allow us to address specific hypotheses. This is a key appeal of volunteered information: some problems of uncertainty become tractable with sufficient observation volume, as we can evaluate the distribution directly instead of relying on sampling methods.
% XXX XXX I'd say we usually avoid volunteered information because of problems of uncertain quality, which we only now are able to work around.
% XXX XXX this whole last paragraph is sort of bleh, it doesn't really *sell* the work. I'm excited! show it here!
