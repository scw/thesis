% State what the data mean, link to question stated in the introduction

% - summarize results/findings
% - put results into context
% - explain implications
% - link to future work + highlight shortcomings

% XXX this paragraph imported from the original introduction, is it better here?
As noted by Goodchild, ``changing technology and economics are moving map production from a system of unified central production to a local patchwork, and the old radial system of dissemination is being replaced with a complex network''\citep{goodchild1999cartographic}. By using the quality assurance methods of record linkage and geographic validation, provides an approach to filter unreliable inputs of shipping data, and take advantage of the newly formed patchwork to broaden our understanding of ocean transportation. The near future will involve global, real-time, high resolution ship data \citep{JonesGoogle2012,carson2012satellite}, but we need methods which accommodate data curation and integration. By using quality assurange methods, I have shown that multiple data dimensions can be incorporated, and heterogenous errors can be rectified.

% This is a key appeal of volunteered information: some problems of uncertainty become tractable with sufficient observation volume, as we can evaluate the distribution directly instead of relying on sampling methods.

Recent calls for increased marine spatial planning at both the national and international level should be met with increased production of fundamental datasets required for effective planning. This work can help advancement of both marine spatial planning and ecosystem-based management, and to help organizations like the IMO provide more effective regulation of shipping, perhaps using insurance-based incentives to reflect environmental costs. The movement toward ubiquitous and real-time data provides an opportunity to greatly improve our management of ocean resources.

This work is a step toward understanging the global effects of shipping. True cost-path movements, which account for vessel preference and barriers, will give us a way of understanding the relative value of different areas within the ocean to the shipping industry. An abstract network model, which incorporates the detailed movement model developed, would allow us to interact with this complex data in a much simpler way, and potentially lead to breakthroughs in marine spatial planning at regional scales. More immediately, these results can feed into understanding the anthropogenic sources of sound in the ocean, and improve models of ship strikes, by providing detailed, holistic speed models for most of the oceans.
% The ship classifications should be derived rigorously, perhaps directly relying on observation data such as those presented here~\citep{adams2011constructing}.

% phantom ships phenomena
