% State what the data mean, link to question stated in the introduction

% - summarize results/findings
% - put results into context
% - explain implications
% - link to future work + highlight shortcomings

The future will include global AIS coverage \citep{JonesGoogle2012,carson2012satellite}, but methods to validate and integrate this data into a scientifically useful system do not yet exist. Here we have shown that by using VGI methods, we can incorporate many data dimensions and correct for multiple forms of error present in the data.

This work leaves much undone. True cost-path movements, accounting for vessel preference and barriers, will give us a way of understanding the relative value of different areas within the ocean to the shipping industry. An abstract network model, which incorporates the detailed movement model developed, would allow us to interact with this complex data in a much simpler way, and potentially lead to breakthroughs in marine spatial planning at regional scales. More immediately, these results can feed into understanding the anthropogenic sources of sound in the ocean, and improve models of ship strikes, by providing detailed, holistic speed data for much of the sea.

Recent calls for increased marine spatial planning at both the national and international level should be met with increased production of fundamental datasets required for effective planning. This work can help advancement of both marine spatial planning and ecosystem-based management. This work can also help organizations like the IMO on more effective regulation of shipping, perhaps using insurance-based incentives to reflect environmental costs.

% The ship classifications should be derived riguously, perhaps directly relying on observation data such as those presented here~\citep{adams2011constructing}.


% phantom ships phenomena
