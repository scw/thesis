\chapter{AIS Applications}
\label{cha:applications}

\begin{itemize}
 \item Marine spatial planning requires better models of human use of the oceans, this work fills in gaps in shipping and fishing activity.
 \item Shipping is a key cause of mortality in many large bodied marine species, both speed and vessel type are important variables which this work provides.
 \item Transportation geography has synthetic global models of movement for vehicles, trains and planes, this work shows usage patterns of the ocean.
 \item Volunteered geographic information uses citizens as sensors, as this dataset is entirely volunteered observations, there are linkages with this emerging field of geography.
\end{itemize}


\subsection{Movement models}

  As in landscape ecology, we need movement models to help predict distributions and trajectories from limited observations. I explore different options for evaluating the movement of ships, including:
\begin{itemize}
 \item brownian bridge
 \item circuit resistance
 \item randomized shortest paths
\end{itemize}

  Based on the constraints of observed speed and course, we can make inference about the trajectories of the ships in the open ocean where we lack direct observation.


\section{Ecological Models}

\textit{This section may be cut depending on time constraints: Ben recommends I nix it and focus on the ocean roads.}

\begin{itemize}
    \item Ecological model \#1: ship strikes. Based on existing work in the Atlantic, show how changes in shipping behavior can dramatically change the population dynamics of the endangered North Atlantic right whales.

    \item Ecological model \#2: Noise pollution. NOAA has recently created a working group to evaluate the impact of noise pollution on marine life, this data greatly improves our understanding of the distribution of shipping. Show how improved data can be linked to better management.

    \item Ecological/economic model: The per-TEU ecological cost model, incorporating our understanding of the environment and human constraints into how we utilize the ocean, based on the differential pricing schemes we already charge for piracy.⋅
     % This last one is exciting, so at least get the ball rolling on it.
\end{itemize}
