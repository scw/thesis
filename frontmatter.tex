  %% Template pages
  \approvalpage 
  \copyrightpage

  %% Dedication
%   \begin{dedication}
%     \null\vfil{%\large
%       \hspace{\stretch{1}}
%       \begin{minipage}{3.2in}
%         \begin{center}
%           Dedicated to a better world.
%         \end{center}
%       \end{minipage}
%     } \vfil\null
%   \end{dedication}

  %% Acknowledgements
  \begin{acknowledgements}
    \addcontentsline{toc}{chapter}{Acknowledgements} 

    No work is the act of one alone, and this thesis is no exception. It would not have been possible without the thoughts and guidance of many. I owe the idea to Ben Halpern, without whom I wouldn't have had worked on interesting marine problems, or even thought twice about shipping. The whole Gaines lab, including Steven Gaines himself, have been tremendously supportive, even when it required rose-colored glasses to see the connections between our work. Robert Warner was gracious enough to join my committee on short notice, and provide insightful context. Krzysztof Janowicz kindly connected this work with the geographic world. The National Center for Ecological Analysis and Synthesis (NCEAS) provided me with the intellectual and technical means to carry out this work, including generous use of its computing resources. NCEAS staff Mark Schildhauer and Jim Regetz helped me see new ways of marrying technology to science. Feedback from NCEAS postdocs, including Jennifer Balch, Jai Rangnathan, Carrie Kappel, and Darren Hardy have left this work richer. NCEAS staff member Ben Best has been a wonderful resource for understanding the ecological context of shipping. Megan McKenna and Steve Katz were gratious enough to share with me an application area, and taught me about how to intepret ship observations. The manuscript is improved thanks to the thoughtful and careful feedback of Oliver Soong. I am indebted to Dawn Wright, for seeing me through the fire, and supporting me through the transition between academia and industry. Thanks to my family and friends, for being there, even when they didn't understand what I was working on, or why anyone would care. And lastly, thanks to my fiancée, Jessica Phillips, who has helped me in too many ways to list.

    \setlength{\epigraphwidth}{.9\textwidth}
    \setlength{\epigraphrule}{0pt}
    \epigraph{\textit{If you want to build a ship, don't drum up the men to gather wood, divide the work and give orders. Instead, teach them to yearn for the vast and endless sea.}}
             {---\textsc{Antoine de Saint-Exupéry}}
  \end{acknowledgements}
\ssp

\dsp

  %% Abstract
  % XXX make this 150 words or less, that's the print limit for this medium.
  \begin{abstract}
    \addcontentsline{toc}{chapter}{Abstract} 
 Shipping, the ocean transportation of people and goods, conveys 90\% of world goods, and plays a crucial role in the global transportation network. Recent work has found climate change, fishing, and shipping as the greatest anthropogenic threats to marine health, but little work has examined the role of shipping.  This work contributes foundational knowledge on the state and distribution of shipping, to help understand this key user of the ocean, and identify key areas where mitigation of impacts are achievable.
% XXX currently 80 words
% ADD: key results, coverage, and why it matters.

% Holistic management of the oceans aimed at preserving the ocean's benefits~\citep{Lubchenco2010} requires detailed quantitative information. 

% Current efforts have emphasized broad abiotic factors such as climate, and biotic factors such as fishing, but little work has examined the role of shipping, despite its importance to both the economy and the environment.  

% a good example abstract:
% In the coming decades, continued population growth, rising meat and dairy consumption and expanding biofuel use will dramatically increase the pressure on global agriculture. Even as we face these future burdens, there have been scattered reports of yield stagnation in the world’s major cereal crops, including maize, rice and wheat. Here we study data from ~2.5 million census observations across the globe extending over the period 1961–2008. We examined the trends in crop yields for four key global crops: maize, rice, wheat and soybeans. Although yields continue to increase in many areas, we find that across 24–39\% of maize-, rice-, wheat- and soybean-growing areas, yields either never improve, stagnate or collapse. This result underscores the challenge of meeting increasing global agricultural demands. New investments in underperforming regions, as well as strategies to continue increasing yields in the high-performing areas, are required.
 
%    \abstractsignature
  \end{abstract}

%%% Local Variables: 
%%% mode: latex
%%% TeX-master: "MAIN"
%%% End: 
