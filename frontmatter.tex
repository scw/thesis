  %% Template pages
  \approvalpage 
  \copyrightpage

  %% Dedication
%   \begin{dedication}
%     \null\vfil{%\large
%       \hspace{\stretch{1}}
%       \begin{minipage}{3.2in}
%         \begin{center}
%           Dedicated to a better world.
%         \end{center}
%       \end{minipage}
%     } \vfil\null
%   \end{dedication}

  %% Acknowledgements
  \begin{acknowledgements}
    \addcontentsline{toc}{chapter}{Acknowledgements} 

    No work is the act of one alone, and this thesis is no exception. This masters project would not have been possible without the inspiration and guidance of many. I owe the foundational ideas of the project to Ben Halpen. Without his advice and encouragement, I may have never worked on interesting marine problems, or even thought twice about shipping. Steven Gaines and the entire Gaines lab have been tremendously supportive even when it required rose-colored glasses to see the connections between our work. Robert Warner was gracious enough to join my committee on short notice, and provide insightful context. Krzysztof Janowicz kindly connected this work with the geographic world. The National Center for Ecological Analysis and Synthesis (NCEAS) provided me with the intellectual and technical means to carry out this work, including generous use of its computing resources. NCEAS staff Mark Schildhauer and Jim Regetz helped me see new ways of marrying technology to science. Feedback from NCEAS postdoctoral researchers, including Jennifer Balch, Jai Rangnathan, Carrie Kappel, and Darren Hardy have left this work richer. NCEAS staff member Ben Best has been a wonderful resource for understanding the ecological context of shipping. Megan McKenna and Steve Katz were generous enough to share with me an application area, and taught me how to intepret ship observations. The manuscript is improved thanks to the thoughtful and careful feedback of Oliver Soong and Jamie Walbridge. I am indebted to Dawn Wright, for seeing me through the fire, and supporting me through the transition between academia and industry. Thanks to my family and friends, for being there, even when they didn't understand what I was working on, or why anyone would care. To Charlie, who has been a great morale booster on many late nights. And lastly, thanks to my fiancée, Jessica Phillips, who has helped me in too many ways to list.

    \setlength{\epigraphwidth}{.9\textwidth}
    \setlength{\epigraphrule}{0pt}
    \epigraph{\textit{If you want to build a ship, don't drum up the men to gather wood, divide the work and give orders. Instead, teach them to yearn for the vast and endless sea.}}
             {---\textsc{Antoine de Saint-Exupéry}}
  \end{acknowledgements}
\ssp

\dsp

  %% Abstract
  % 150 words or less, that's the print limit for this medium.
  % "We no longer have a word limit on your abstract, as this constrains your ability to describe your research in a section that is accessible to search engines, and therefore would constrain potential exposure of your work. However, we continue to publish print indexes that include citations and abstracts of all dissertations and theses published by ProQuest/UMI  print indexes require limits of 350 words for doctoral dissertations and 150 words for master's theses."
  \begin{abstract}
    \addcontentsline{toc}{chapter}{Abstract}
     
Shipping, the ocean transportation of people and goods, moves most world trade, and understanding its effects is required to assess human use of the oceans.  This work examines the shipping trade by combining global observations of ship location with vessel identification records, and interpreting the results in an ecological context. By incorporating quality checking methods with volunteered geographic information, I provide a spatially resolved high resolution dataset which links individual vessels with their movement patterns and attributes. This contributes knowledge on the state and distribution of shipping, and identifies areas where mitigation of impacts are achievable.

% XXX currently 106 words
% ADD: key results, coverage, and why it matters.
%This is the first work to combine extensive global ship observations~\citep{VOSClim,Tetreault2002} with vessel identification records to both address questions on ecological effects of shipping, and to expand our understanding of the global shiping trade. Most previous works focused on regional problems, and the limited global work available used poor data~\citep{Corbett2007, Halpern2008}. Incorporating geographic quality checking methods~\citep{goodchildli2012} with volunteered geographic information (VGI, \citealp{goodchild2007citizens}), I provide a spatially resolved high resolution dataset which links individual ships with their movement patterns and vessel attributes. This dataset, when aggregated into multiple representations, allows us to answer open questions on topics including noise pollution, ship strikes, and invasive species introduction. 


%The future will include global AIS coverage \citep{JonesGoogle2012,carson2012satellite}, but methods to validate and integrate this data into a scientifically useful system do not yet exist. Here we have shown that by using VGI methods, we can incorporate many data dimensions and correct for multiple forms of error present in the data.
% Holistic management of the oceans aimed at preserving the ocean's benefits~\citep{Lubchenco2010} requires detailed quantitative information. 

% Current efforts have emphasized broad abiotic factors such as climate, and biotic factors such as fishing, but little work has examined the role of shipping, despite its importance to both the economy and the environment.  

    %\abstractsignature
  \end{abstract}

%%% Local Variables: 
%%% mode: latex
%%% TeX-master: "MAIN"
%%% End: 
