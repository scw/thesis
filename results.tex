\section{\textbf{Results}}
% A concise report of your results. A final report is not a lab log book, so you do not need to include details of every calculation or include every graph.  Most readers get bored if there are too many very similar graphs in a paper. The key requirement is to report all findings that relate directly to your goals and that will be referred to in the discussion.  

% Avoid having the text in the Results section be a simple list of each result or statistical test – integrate, summarize, synthesize, and point out the most important messages you want your readers to take away from the paper.

% What really matters from what we've done?

\subsection{Record Linkage}

% XXX move this table to an appendix? doesn't really help tell the story.
% table describing # of records linked betwen each source
% SOURCE: ship-id-model/matches.ods
\begin{table}[htbp]
  \begin{tabular}{rrll} %{\centering\arraybackslash}p{2cm}>{\centering\arraybackslash}p{5cm}>{\centering\arraybackslash}p{9cm}}
    \hline
    Source $A$ & Source $B$ & matched records & \% possible matched \\
    \hline
     DS & FCC &  3481 & 50.35\textsuperscript{1} \\
     DS & ITU & 41380 & 30.73 \\
     DS &  VT & 72286 & 53.68 \\
    FCC & ITU & 27874 & 50.58\textsuperscript{1} \\
    FCC &  VT &  5282 & 53.23\textsuperscript{1} \\
     VT & ITU & 54727 & 43.25 \\
  \end{tabular}
  \caption{Ship record linkage results summary. \\
    1. FCC data is US only, \% possible is of US-only data from each source.}
  \label{table:ships-record-linkage-results-summary}
\end{table}


\subsection{Geographic Validation}

% XXX: how did we improve data quality by using geographic filtering? what does that tell us, and why does it matter?


\subsection{Ship results}
% Include our graphs: each of our ship types, with two sets of plots: ship density, and ship speed.

% XXX for now: stub the graphs we intend to include: maps of ship movement, classified by TYPE as derived from our linked observations.

% graphs of SHIP SPEED by type, either using kernel estimation or simple histograms depending on the clarity of the data.
\begin{figure}[htbp]
  \centering
  \includegraphics[width=120mm]{figures/speed-boxplot.pdf}
  \caption{Boxplot of validated ship speeds.}
  \label{fig:vessel-speed-boxplot}
\end{figure}

\begin{figure}[htbp]
  \centering
  \includegraphics[width=180mm]{figures/speed-comparison-qqplot.pdf}
  \caption{Distribution of validated ship movement speeds, computed using kernel density comparisons \cite{bowman1997applied}}
  \label{fig:vessel-speed-density}
\end{figure}

% XXX include our VALIDATED SHIP MAPS
% For now: doing a log transformation to merge the sailwx and AIS data. 

For each cell the output density value as calculated as a log-transformed equal-weighted addition between the two input datasets, AIS and VOS:
\begin{equation}
 x = e^{\ln(R_{AIS}) + \ln(R_{VOS})}
\end{equation}

Average speed per cell was calculated as sum speed over all observations $R_{AIS} \cup R_{VOS} = n$, and dividing it by the total number of observations: 
\begin{equation}
 \bar{s} = \frac{\sum\limits_{i=0}^n s}{x}
\end{equation}


Geographic features tend to be clustered, exhibiting spatial autocorrelation. Here, we use Moran's I to compute global autocorellation statistics for our density rasters:
\begin{equation}
I = \frac{n}{\sum_{i=1}^{n}\sum_{j=1}^{n}w_{ij}}
\frac{\sum_{i=1}^{n}\sum_{j=1}^{n}w_{ij}(x_i-\bar{x})(x_j-\bar{x})}{\sum_{i=1}^{n}(x_i - \bar{x})^2}
\end{equation}

where $n$ is the number of cells, and $w_{ij}$ is the spatial weight.


% XXX: suggestion-- ships by season? just static results here, can do some dynamic stuff later.
